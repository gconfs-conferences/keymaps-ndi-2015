\documentclass[c,12pt]{beamer}

\usepackage[T1]{fontenc} 
\usepackage[utf8]{inputenc}
\usepackage[frenchb]{babel}

% readable pdf
\usepackage{pslatex}
\usepackage{alltt}
\usepackage{verbatim}
\usepackage{moreverb}

\usepackage{graphicx}
\usepackage{animate}
\usepackage{xmpmulti}

\usetheme[secheader]{Madrid}
\usecolortheme{whale}
\useinnertheme[shadow]{rounded}
\useoutertheme{infolines}

\makeatother
\setbeamertemplate{footline}
{
  \leavevmode%
  \hbox{%
  \begin{beamercolorbox}[wd=.4\paperwidth,ht=2.25ex,dp=1ex,center]{author in head/foot}%
    \usebeamerfont{author in head/foot}\insertshortauthor
  \end{beamercolorbox}%
  \begin{beamercolorbox}[wd=.6\paperwidth,ht=2.25ex,dp=1ex,center]{title in head/foot}%
    \usebeamerfont{title in head/foot}\insertshorttitle\hspace*{3em}
    \insertframenumber{} / \inserttotalframenumber\hspace*{1ex}
  \end{beamercolorbox}}%
  \vskip0pt%
}

% contenu de la page de titre
\title{La suprématie des claviers ERTY : pourquoi et comment lutter ?}
\subtitle{(et surtout : combien ça coûte ?)}
\author{Meven \texttt{'mevouc'} \bsc{Courouble}}
\date{\oldstylenums{3 décembre 2015}}

\logo{\includegraphics[height=0.5cm]{GConfs.png}}

\AtBeginSection[]{
	\begin{frame}{Sommaire}
		\tableofcontents[currentsection, hideothersubsections]
	\end{frame}
}



\begin{document}

\frame{\titlepage}

\begin{frame}
	\frametitle{Sommaire}
	\tableofcontents[hideallsubsections]
\end{frame}

\section{Pour commencer}
\subsection{Historique}

\begin{frame}
	\frametitle{\subsecname}
	\begin{itemize}
		\item<1-> \textbf{XIX\ieme siècle :} Développement de la machine à écrire
			\begin{itemize}
				\item<2-> Nécessité de normaliser la position des touches, dispositon
					alphabétique premièrement adoptée
				\item<3-> Apparition des dactylographes : Rapidité, précision
			\end{itemize}
		\item<4-> \textbf{1873 :} Disposition QWERTY
			\begin{itemize}
				\item<5-> Pallier à des contraintes mécaniques sur les machines à écrire
				\item<6-> AZERTY déposé une vingtaine d'années plus tard
			\end{itemize}
		\item<7-> \textbf{1930 :} Dvorak, apparition de dispositions alternatives
		\item \textbf{Années 1960 :} Premiers claviers d'ordinateurs
	\end{itemize}
\end{frame}

\subsection{Terminologie}

\begin{frame}
	\onslide<1->
	\frametitle{\subsecname}
	\onslide<2->
	\begin{block}{Clavier}
		Périphérique physique permettant la saisie textuelle.
	\end{block}
	\onslide<3->
	\begin{block}{Disposition (Keymap)}
		Interprétation logique de chaque touche par l'OS, le logiciel, etc.
	\end{block}
\end{frame}

\section{Stop aux ERTY : pourquoi ?}
\subsection{Inadaptés à l'usage qu'on en a}

\begin{frame}
	\onslide<1->
	\frametitle{\subsecname}
	\begin{itemize}
		\item<2->{Ne contiennent pas tous les caractères}
		\item<3->{Touches mal réparties}
	\end{itemize}
\end{frame}

\subsection{Provoquent une frappe plus lente}

\begin{frame}
	\frametitle{\subsecname}
	\begin{itemize}
		\item<1->{Touches trop éloignées (\texttt{Enter}, \texttt{Esc})}
		\item<2->{Aucune logique dans la disposition des lettres}
		\item<3->{Dispositions classiques incitent à regarder son clavier}
	\end{itemize}
\end{frame}

\subsection{Autres raisons}

\begin{frame}
	\frametitle{\subsecname}
	\begin{itemize}
		\item<1->{Mauvaise position et mauvais usage des doigts et des mains}
			\begin{itemize}
				\item{Peut provoquer à terme des problèmes d'arthrose}
			\end{itemize}
		\item<2->{Fierté personnelle \texttt{:D}}
	\end{itemize}
\end{frame}

\section{Stop aux ERTY: comment ?}
\subsection{Changer de keymap}

\begin{frame}
	\frametitle{\subsecname}
	\begin{itemize}
		\item{C'est gratuit !}
		\item<2->{Une keymap adaptée à l'utilisation de son clavier (coder,
			rédiger,etc.)}
	\end{itemize}
\end{frame}

\begin{frame}
	\frametitle{Dvorak}
	\begin{center}
		\includegraphics[scale=0.26]{Dvorak.png}
	\end{center}
\end{frame}

\begin{frame}
	\frametitle{\subsecname}
	\begin{itemize}
		\item<1->{C'est gratuit !}
		\item<1->{Une keymap adaptée à l'utilisation de son clavier (coder,
			rédiger,etc.)}
		\item<2->{Une keymap adaptée à sa langue d'écriture}
	\end{itemize}
\end{frame}

\begin{frame}
	\frametitle{Dvorak-fr}
	\begin{center}
		\includegraphics[scale=0.26]{Dvorak-fr.png}
	\end{center}
\end{frame}

\begin{frame}
	\frametitle{Bépo}
	\begin{center}
		\includegraphics[scale=0.26]{Bepo.png}
	\end{center}
\end{frame}

\begin{frame}
	\frametitle{\subsecname}
	\begin{itemize}
		\item{C'est gratuit !}
		\item{Une keymap adaptée à l'utilisation de son clavier (coder,
			rédiger,etc.)}
		\item<1->{Une keymap adaptée à sa langue d'écriture}
		\item<2->{Son propre mapping}
			\begin{itemize}
				\item<3->{Swap \texttt{CapsLock} et \texttt{Esc}}
				\item<4->{Utiliser des macros}
				\item<5->{Touche morte (touche \texttt{Compose})}
			\end{itemize}
	\end{itemize}
\end{frame}

\subsection{Changer de clavier}

\begin{frame}
	\frametitle{\subsecname}
	\begin{itemize}
		\item<1->{C'est pas gratuit...}
		\item<2->{Claviers mécaniques}
		\item<3->{Claviers matriciels}
		\item<4->{Claviers ergonomiques}
	\end{itemize}
\end{frame}

\begin{frame}
	\frametitle{Interrupeur à membrane}
	\begin{center}
		\includegraphics[scale=0.32]{membrane.png}
	\end{center}
\end{frame}

\begin{frame}
	\frametitle{Interrupteurs mécaniques}
	\begin{center}
		\includegraphics[scale=0.2]{meca.png}
	\end{center}
\end{frame}

\begin{frame}
	\frametitle{Clavier ergonomique matriciel : le TypeMatrix}
	\begin{center}
		\includegraphics[scale=0.32]{typematrix.jpg}
	\end{center}
\end{frame}

\begin{frame}
	\frametitle{Clavier ergonomique scindé}
	\begin{center}
		\includegraphics[scale=2.5]{scinde.jpg}
	\end{center}
\end{frame}

\begin{frame}
	\frametitle{Clavier ergonomique ultime : le Maltron}
	\begin{center}
		\includegraphics[scale=0.35]{maltron.jpg}
	\end{center}
\end{frame}

\begin{frame}
	\frametitle{Clavier ergonomique ultime : le Kinesis}
	\begin{center}
		\includegraphics[scale=0.35]{kinesis.jpg}
	\end{center}
\end{frame}

\subsection{Changer ses habitudes}

\begin{frame}
	\frametitle{\subsecname}
	\begin{itemize}
		\item<1->{C'est gratuit !}
		\item<2->{Utiliser les repères sous les index (touches \texttt{F} et
			\texttt{J})}
		\item<3->{Utiliser tous ses doigts}
		\item<4->{Éviter le pavé numérique !}
		\item<5->{Utiliser les 2 \texttt{Shift}, les 2 \texttt{Alt}, les 2
			\texttt{Ctrl}}
		\item<6->{Utiliser des logiciels et programmes avec des bindings pratiques}
		\item<7->{Configurer soi-même ses bindings}
	\end{itemize}
\end{frame}

\subsection{Apprendre}

\begin{frame}
	\frametitle{\subsecname}
	\begin{itemize}
		\item<1->{C'est gratuit !}
		\item<2->{Test.s en ligne (\url{http://10fastfingers.com})}
		\item<3->{Avoir du temps libre}
		\item<4->{Ne pas avoir peur d'être lent les premiers jours}
	\end{itemize}
\end{frame}

\begin{frame}
	\onslide<1->
	\frametitle{\subsecname}
	\begin{center}
		\onslide<2->
		\includegraphics[scale=0.2]{mandela.jpg}
	\end{center}
\end{frame}

\begin{frame}
	\frametitle{Merci}
	\begin{center}
		\emph{Questions, remarques, choses pas claires ?}
	\end{center}
\end{frame}

\end{document}

% Pour commencer
%		Historique
%		Terminologie
% Stop aux ERTY, pourquoi ?
%		Lents
%		Incite à regarder le clavier
%		Ne contient pas tous les caractères
%		Certaines touches trop éloignées (Enter, Esc)
%		Mauvaise position des doigts et des mains -> Arthrose doigts et mains
%		Fierté personnelle :D
% Comment ?
%		Changer de Keymap
%			Dvorak
%			Dvorak-fr
%			Bépo
%			Son propre mapping : CapsLock-Escape, Compose
%		Changer de clavier
%			Clavier méca > all
%			claviers matriciels
%			claviers ergonomiques (scindés, inclinés, etc.)
%		Changer ses habitudes
%			Utiliser les repères sur les touches centrales
%			Utiliser tous ses doigts
%			Stop au pavé numérique -> trop loin
%			Utiliser les 2 shifts, les 2 alt, les 2 Ctrl
%			Bonus : Refaire ses bindings dans ses logiciels favoris
%			Utiliser des logiciels avec des bindings pratiques
%		L'apprentissage
%			Tests de dactylographie
%			Avoir un peu de temps libre
%			Ne pas avoir peur d'être lent au début
