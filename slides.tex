\documentclass[c,12pt]{beamer}
%
% Packages pour le français
\usepackage[T1]{fontenc} 
\usepackage[utf8]{inputenc}
\usepackage[frenchb]{babel}
%
% pour un pdf lisible à l'écran
% il y a d'autres choix possibles 
\usepackage{pslatex}
%
% pour le style et couleurs
\usetheme[secheader]{Madrid}
\usecolortheme{whale}
\useinnertheme[shadow]{rounded}
\useoutertheme{infolines}

\makeatother
\setbeamertemplate{footline}
{
  \leavevmode%
  \hbox{%
  \begin{beamercolorbox}[wd=.4\paperwidth,ht=2.25ex,dp=1ex,center]{author in head/foot}%
    \usebeamerfont{author in head/foot}\insertshortauthor
  \end{beamercolorbox}%
  \begin{beamercolorbox}[wd=.6\paperwidth,ht=2.25ex,dp=1ex,center]{title in head/foot}%
    \usebeamerfont{title in head/foot}\insertshorttitle\hspace*{3em}
    \insertframenumber{} / \inserttotalframenumber\hspace*{1ex}
  \end{beamercolorbox}}%
  \vskip0pt%
}
%
% contenu de la page de titre
\title{La suprématie des claviers ERTY, pourquoi et comment lutter ?}
%\subtitle{How to keyboard}
\author{Meven "mevouc" \bsc{Courouble}}
\date{\oldstylenums{December 2015}}

\logo{\includegraphics[height=0.5cm]{GConfs.png}}

\AtBeginSection[]{
	\begin{frame}{Sommaire}
		\tableofcontents[currentsection, hideothersubsections]
	\end{frame}
}

%
% Fin du préambule
%

\begin{document}

\frame{\titlepage}

\section{Pour commencer}
\subsection{Historique}

\begin{frame}
	\frametitle{\subsecname}
	\begin{itemize}
		\item<1-> Début XIX\ieme siècle : Développement de la machine à écrire
			\begin{itemize}
				\item<2-> Nécessité de normaliser la position des touches
				\item<3-> Apparition des dactylographes : Rapidité, précision
			\end{itemize}
		\item<4-> 1873 : Disposition QWERTY
			\begin{itemize}
				\item<5-> Pallier à des contraintes mécaniques 
				\item<6-> AZERTY déposé une vingtaine d'années plus tard
			\end{itemize}
		\item<7-> 1930 : Dvorak, apparition de dispositions alternatives
	\end{itemize}
\end{frame}

\subsection{Terminologie}

\begin{frame}
	\onslide<1->
	\frametitle{\subsecname}
	\onslide<2->
	\begin{block}{Clavier}
		Périphérique physique permettant la saisie textuelle.
	\end{block}
	\onslide<3->
	\begin{block}{Disposition (Keymap)}
		Interprétation logique de chaque touche par l'OS, le logiciel, etc.
	\end{block}
\end{frame}


\end{document}

% Pour commencer
%		Historique
%		Terminologie
% Stop aux ERTY, pourquoi ?
%		Lents
%		Incite à regarder le clavier
%		Ne contient pas tous les caractères
%		Certaines touches trop éloignées (Enter, Echap)
%		Mauvaise position des doigts et des mains -> Arthrose doigts et mains
%		Fierté personnelle :D
% Comment ?
%		Changer de Keymap
%			Dvorak
%			Dvorak-fr
%			Bépo
%			Son propre mapping : CapsLock-Escape, Compose
%		Changer de clavier
%			Clavier méca > all
%			claviers matriciels
%			claviers ergonomiques (scindés, inclinés, etc.)
%		Changer ses habitudes
%			Utiliser les repères sur les touches centrales
%			Utiliser tous ses doigts
%			Stop au pavé numérique -> trop loin
%			Utiliser les 2 shifts, les 2 alt, les 2 Ctrl
%			Bonus : Refaire ses bindings dans ses logiciels favoris
%			Utiliser des logiciels avec des bindings pratiques
